\begin{abstract}
  This thesis is about using statistical methods for performance and power estimation which would allow us to develop better
  scheduling algorithms and also more energy efficient systems.
  In many deployments, computer systems are underutilized – meaning that
  applications have performance requirements that demand less than full
  system capacity. Ideally, we would take advantage of this under-utilization
  by allocating system resources so that the performance requirements are met
  and energy is minimized. This optimization problem is complicated by the fact
  that the performance and power consumption of various system configurations
  are often application – or even input – dependent. Thus, practically,
  minimizing energy for a performance constraint requires fast, accurate
  estimations of application-dependent performance and power tradeoffs. We propose
  a set of algorithms for different scenarios to tackle this problem.
First, we propose LEO, a probabilistic graphical model-based learning system that
  provides accurate online estimates of an application’s power and performance
  as a function of system configuration. This work mostly focuses on the performance estimation
  for single applications. As the second part of our work, we design a system called CALOREE which allows
  the learnt models to be combined with a controller so that the system is robust
  to dynamic situations with changing resource requirement. Finally, as the third
  part of our work, we look into the estimation for application’s performance when
  they are co-scheduled with other applications.  Applications co-scheduled on the same physical hardware interfere with one
  another by contending for shared resources. Predicting this interference ahead of time would be particularly valuable for
  job scheduling. We therefore propose an efficient technique for estimating application interference
  based on sparse regression. We call our approach ESP for Estimating co-Scheduled Performance.

  LEO uses a graphical model to integrate a small number of observations of the current application with
knowledge of the previously observed applications to produce accurate estimations of
power and performance trade-offs for the current application in all configurations.
LEO produces the most accurate estimates and near optimal energy savings.
These estimates can greatly resource allocation in static situation.
But the second major challenge in real systems is \textit{dynamics}, dynamics—performance must
be maintained despite unpredictable changes in operating environment or input.
Machine learning accurately predicts the performance of complex, interacting resources, but does not address
system dynamics; control theory adjusts resource usage dynamically, but
   struggles with complex resource
   interaction. We therefore propose CALOREE, a combination of learning and control
   that automatically adjusts
   resource usage to meet performance requirements with minimal energy in complex, dynamic environments.
   CALOREE breaks resource allocation into two sub-tasks: learning speedup as a function of resource usage, and
   controlling speedup to meet performance requirements. CALOREE also defines a general interface allowing
   different learners to be combined with a controller while maintaining control’s formal guarantees that
   performance will converge to the goal. We implement CALOREE and test its ability to deliver reliable
   performance on heterogeneous ARM big.LITTLE architectures in both single and multi-application scenarios.
   Compared to state-of-the-art learning and control solutions, we find that CALOREE reduces deadline misses
   by 2–6x while reducing energy consumption by 7–10\%.
Finally, the additional challenge that real systems face is performance loss due
to application interference. We quantify interference as slowdown, or the performance loss one application
   experiences in the presence of co-scheduled applications.
    Given an accurate interference prediction, a scheduler can determine
   optimal assignments of applications to physical machines, leading to higher throughput
   in batch systems and better quality-of-service for latency-sensitive applications.
   In data centers and super computers schedulers often have a great deal of accumulated
   data about past jobs and their interference, yet turning this data into effective
   interference predictors is difficult.
   We explore such state-of-the-art regularized regression models for estimating
   application interference. We find that regularized linear regression methods
   require a relatively small number of features, but produce inaccurate models.
   In contrast, non-linear models that include interaction terms – i.e., permit
   features to be multiplied together – are more accurate, but are extremely inefficient
   and not practical for online scheduling.
The key insight in ESP is to split regression modeling into two parts: feature
   selection and model building. ESP uses linear techniques to perform feature selection,
   but uses quadratic techniques for model building. The result is a highly accurate
   predictor that is still practical and can be integrated into a real application scheduler.
\end{abstract}
