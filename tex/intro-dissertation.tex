\section{Introduction}

% Dennard Scaling making energy essential.  Architects address energy
% by making more complicated processors which expose resources to
% software management.  For a wide range of applications, need to meet
% performance goals with minimal energy.
Large classes of computing systems---from embedded to cloud---must
deliver reliable performance to users while minimizing energy to
increase battery life or decrease operating costs.  To address these
conflicting requirements, hardware architects expose diverse,
heterogeneous resources with a wide array of performance and energy
tradeoffs.  Software must allocate these resources to
guarantee performance requirements are met with minimal energy. Energy is increasingly important; reducing energy consumption reduces
operating costs in datacenters and increases battery life in mobile
devices.  Second, computer systems are often underutilized, meaning
there are significant portions of time where application performance
demands do not require the full system capacity
\cite{google2007,MeisnerISCA2011}.

% Problem: How to minimize energy while meeting load demands
These trends raise the problem of allocating available resources
to meet the current performance demand while minimizing energy
consumption. This problem is a \emph{constrained optimization}
problem. The current utilization level represents a performance
constraint (\ie an amount of work that must be completed in a given
time); system energy consumption represents the objective function to
be minimized. There are three primary difficulties in determining how to allocate
heterogeneous resources.

The first is \emph{complexity}: resources
interact in complicated ways, leading to non-convex optimization
spaces. This problem is challenging because it requires a great deal of
knowledge to solve.  More than knowledge of the single fastest, or
most energy efficient system configuration solving this problem
requires knowledge of the power and performance available in all
configurations and the extraction of those configurations that
represent Pareto-optimal tradeoffs.  Acquiring this knowledge is
additionally complicated by the fact that these power/performance
tradeoffs are often application -- or even input -- dependent.  Thus,
there is a need for techniques that accurately estimate these
application-dependent parameters during run-time.

The second is \emph{dynamics}: perfor\-mance requirements
must be met despite unpredictable disturbances; \eg{} phases in input
or changes in operating environment.  Prior work addresses each of
these difficulties individually. We want the benefits of both learning and control
to ensure
performance requirements are met with minimal energy in complex and
dynamic environments.

The third is \emph{application interference}: Applications co-scheduled on the
same physical hardware \emph{interfere} with one another by contending for shared resources
\cite{dwyer2012practical,kambadur2012measuring,Bubble-flux,merkel2010resource}. By accurately predicting this interference, a scheduler
can determine optimal assignments of applications to physical
machines, leading to higher throughput in batch systems and better
quality-of-service for latency-sensitive applications. Data center and super computer operators often have a great deal of
accumulated data about past jobs and their interference, yet turning
this data into effective interference predictors is difficult
\cite{kambadur2012measuring}.

In this thesis we present new statistical methods for performance and power estimation which would allow us to develop better scheduling algorithms and also more energy efficient systems.



\section{Background}
Machine learning techniques represent a promising approach to
addressing this estimation problem.  \emph{Offline learning}
approaches collect profiling data for known applications and use that
to predict optimal behavior for unseen applications (examples include
\cite{Yi2003,Koala,LeeBrooks2006,CPR,ChenJohn2011}).  \emph{Online
 learning} approaches use information collected while an application
is running to quickly estimate the optimal configuration (examples
include
\cite{Li2006,Flicker,ParallelismDial,Ponamarev,petabricksDynamic,LeeBrooks,TAAS}).
Offline methods require minimal runtime overhead, but suffer because
they estimate only trends and cannot adapt to particulars of the
current application.  Online methods customize to the current
application, but cannot leverage experience from other applications.
In a sense, offline approaches are dependent on a rich training set
that represents all possible behavior, while the online approaches
generate a statistically weak -- \ie inaccurate -- estimator due to
small sample size.

Many machine learning approaches model the complex performance/power
tradeoff spaces inherent to heterogeneous computing
\cite{reddiHPCA2013,dubach2010,Bitirgen2008,Ipek,Koala,LEO,Flicker,Ponamarev,Paragon}.
Learning handles non-convexity, identifying local optima and moving to
globally optimal solutions. These techniques are computationally
expensive and lack support for dynamics; \ie{} when the environment
changes the expensive model building process must be restarted.
Control theoretic solutions provide formal guarantees that they will
converge to the desired performance despite system dynamics
\cite{Hellerstein2004a,Chen2011,POET,ControlWare,Agilos,grace2,JouleGuard}.
Control provides formal guarantees of convergence to the desired
performance, but these guarantees require \emph{ground truth} models
and control will not converge if the models do not accurately capture
system complexity---including local optima and non-linearities.

  To apply machine learning to build an
accurate predictor from this data, two fundamental decisions must be
made: (1) what \emph{features} should be measured and (2) what
\emph{model} maps these features into an accurate prediction.  Smaller
feature spaces provide more computationally efficient models, but
may miss key data and reduce prediction accuracy.  The art to modeling
is managing the tradeoffs between feature set size and the model
accuracy.  One family of machine learning
techniques---\emph{regularization}--- addresses the particular problem
where the number of features is much larger than the number of
samples; \ie the problem is \emph{ill-posed} and unsolvable with
standard regression analysis.  Regularization methods solve such
ill-posed problems by simultaneously selecting both the features and
the model
\cite{hoerl1988ridge,tibshirani1996regression,zou2005regularization}.


\section{Contribution}
In this work, we present

\begin{itemize}
  \item \SYSTEMLEO{} (Learning for Energy
  Optimization), a learning framework that combines the best of both
  worlds, \ie the statistical properties both offline and online
  estimation.
 \item \SYSTEM{} creates a general control system implementation that factors
 out the parameters that must be learned.
 \item ESP (\textbf{E}stimating co-\textbf{S}cheduled \textbf{P}erformance) for
 application interference estimation.
\end{itemize}

\subsection{\SYSTEMLEO{}}
\SYSTEMLEO{} uses a graphical model to integrate a small number of
observations of the current application with knowledge of the
previously observed applications to produce accurate estimations of
power and performance tradeoffs for the current application in all
configurations.  \SYSTEMLEO{}'s strength is that it quickly matches the
behavior of the current application to a subset of the previously
observed applications.  For example, if \SYSTEMLEO{} has previously seen
an application that only scales to 8 cores, it can use that
information to quickly determine if the current application will be
limited in its scaling.


% Limitations and uses \TODO{Need to have a paragraph here on
%   limitations - but keep it positive.}
\SYSTEMLEO{} is a fairly general approach in that it supports many types
of applications with different resource needs.  It is not, however,
appropriate for all computer systems, especially ones, which run many
small, unique jobs.  Instead, it focuses on supporting systems that 1)
execute longer running jobs (in the 10s of seconds) or many repeated
instances of short jobs, 2) run at a wide range of utilizations, and
3) might have phases where optimal tradeoffs may change online.  For
systems that meet these criteria, \SYSTEMLEO{} provides a powerful
ability to reduce the energy consumption.  For systems that service
short ( $< 1$ second), largely unique jobs, \SYSTEMLEO{} will work, but
other approaches are probably better matched to those specific needs.
%\TODO{its probably alright for full utilization systems too because it can tell which configuration provides maximum utilization. Also, race-to-idle is not mentioned in intro?}

% Our results
We have implemented \SYSTEMLEO{} on a Linux x86 server and tested its
ability to minimize energy for 25 different applications from a
variety of different benchmark suites.  We first compare \SYSTEMLEO{}'s
performance and power prediction accuracy to (1) the true value, (2)
an offline approach, and (3) an online approach (See
\Secref{sec:poc}).  On average, \SYSTEMLEO{} is within 97\% of the true
value while the offline and online approaches only achieve 79\% and
86\% accuracy, respectively.  We then use \SYSTEMLEO{} to minimize energy
for various performance requirements (or system utilizations) (See
\Secref{sec:experiment:LP}).  Overall we find that our approach is
within 6\% of the true optimal energy, while the offline approach
exceeds optimal energy consumption by 29\% and online approach by
24\%.  Finally, we show that \SYSTEMLEO{} provides near optimal energy
savings when adapting to phases within an application.

% Contributions of this work
This work makes the following contributions:
\begin{itemize}
\item To the best of our knowledge, this is the first application of
 probabilistic graphical models for solving crucial system
 optimization problems such as energy minimization.
\item It presents a graphical model capable of accurately
 estimating the application-specific performance and power of
 computer system configurations without prior knowledge of the
 application. (See \Secref{sec:HBN}).
\item It makes the source code for this learning system available in
 both Matlab and C++\footnote{leo.cs.uchicago.edu}.
\item It evaluates \SYSTEMLEO{} on a real system. (See
 \Secref{sec:experiment}).
\item It compares the accuracy of \SYSTEMLEO{}'s estimations to both the
 truth and to offline and online learning approaches (See
 \Secref{sec:experiment:PP}).
\item It integrates \SYSTEMLEO{} into a runtime for energy optimization
 and finds this learning framework achieves near-optimal energy
 savings.  Furthermore, \SYSTEMLEO{} significantly reduces energy
 compared to both offline and online approaches as well as the
 popular race-to-idle heuristic.  (See
 \Secref{sec:experiment:LP}).
\end{itemize}

\subsection{\SYSTEM{}}
\SYSTEM{} creates a general control system implementation that factors
out the parameters that must be learned.  While traditional control
designs assume all combinations of resource configurations have been
measured accurately (requiring 100s or 1000s of samples
\cite{sysid,FSE2015}), \SYSTEM{} tunes its internal models and
parameters while sampling only a small subset of possible
configurations.  Among these, the controller's \emph{pole} is a key
parameter.  \SYSTEM{} automatically tunes the pole, providing formal
guarantees the controller converges to the desired performance---even
when behavior is estimated by a noisy learner rather than directly
measured.  \SYSTEM{} implements modular abstractions and self-tuning
mechanisms such that the expensive learning can be offloaded to
another system and the controller itself runs in
constant---$O(1)$---time.  Thus, learning's high cost can be amortized
as \SYSTEM{} uses \emph{transfer learning} to apply knowledge from
multiple devices and applications \cite{pan2010survey}.

\SYSTEM{}'s generality allows a wide range of learning techniques to
be paired with its controller.  So, \SYSTEM{} not only provides an
advantage over existing individual learning and control techniques, it
allows us to explore different combinations of learning and control to
find the best.

% Implement CALOREE.  Test against state of the art learning and
% self-tuning control systems.  We find that:
We evaluate \SYSTEM{} by implementing the learner on an x86 server and
the controller on heterogeneous ARM big.LITTLE devices.  We compare
\SYSTEM{} to existing, state-of-the-art learning (including polynomial
regression \cite{Koala,dubach2010}, the Netflix algorithm
\cite{netflix,Paragon}, and a hierarchical Bayesian model \cite{LEO})
and control (including proportional-integral-derivative
\cite{Hellerstein2004a} and adaptive, or self-tuning
\cite{HandbookControl}) techniques.  \PUNT{Additionally, we compare to
  a naive combination of learning and control that does not account
  for the inaccuracies of learned models.}  We set performance
goals---as latency requirements---for a set of benchmark applications
and then measure both the percentage of time the requirements are
violated and the energy for each application.  We test both
\emph{single-app}, where an application runs alone, and
\emph{multi-app} environments, where other applications enter the
system and compete for resources.  \SYSTEM{} achieves the:
\begin{itemize}[leftmargin=1em]
\item \textit{Most reliable performance:}
  \begin{itemize}[leftmargin=1em]
  \item In the \emph{single-app} case, the best prior technique misses
    12\% of deadlines on average, while \SYSTEM{} misses only 5\% on
    average---reducing deadline misses by more than 58\% compared to
    prior approaches.
  \item In the \emph{multi-app} case, the best prior approach averages
    30\% deadline misses, but \SYSTEM{} misses just 5.6\% of
    deadlines---a huge improvement over prior work.
  \end{itemize}
\item \textit{Best energy savings:} We compare to an \emph{oracle}
  with a perfect model of the application, system, and future events.
  \begin{itemize}[leftmargin=1em]
  \item In the \emph{single-app} case, the best prior approach
    averages 12\% more energy consumption than the oracle, but
    \SYSTEM{} consumes only 5\% more.
  \item In the \emph{multi-app} case, the best prior approach averages
    18\% more energy than the oracle, while \SYSTEM{} consumes just
    8\% more.
  \end{itemize}
\end{itemize}

% Key contributions.
% Contributions, but I decided against bulleted llist for thsi work
In summary, control approaches are well suited to dynamic environments
and learning techniques accurately model complex, heterogeneous
processors.  \emph{To the best of our knowledge, \SYSTEM{} is the
  first work to combine the two to ensure application performance
  without prior knowledge of the controlled application.}  Our formal
analysis of \SYSTEM{}'s convergence despite noisy inputs shows how to
incorporate learned variance into control theoretic guarantees.  We
demonstrate the practical benefits of these contributions by
implementing \SYSTEM{} for mobile/embedded processors to show it
provides more reliable performance and lower energy than individual,
state-of-the-art learning and control solutions.

\subsection{\SYSTEMESP{}}
To apply machine learning to build an
accurate predictor from this data, two fundamental decisions must be
made: (1) what \emph{features} should be measured and (2) what
\emph{model} maps these features into an accurate prediction.  Smaller
feature spaces provide more computationally efficient models, but
may miss key data and reduce prediction accuracy.  The art to modeling
is managing the tradeoffs between feature set size and the model
accuracy.  One family of machine learning
techniques---\emph{regularization}--- addresses the particular problem
where the number of features is much larger than the number of
samples; \ie the problem is \emph{ill-posed} and unsolvable with
standard regression analysis.  Regularization methods solve such
ill-posed problems by simultaneously selecting both the features and
the model
\cite{hoerl1988ridge,tibshirani1996regression,zou2005regularization}.

This work explores such state-of-the-art regularization models for
predicting application interference.  We find that regularized linear
regression methods require a relatively small number of features, but
produce inaccurate models.  In contrast, non-linear models that
include \emph{interaction terms}---\ie permit features to be
multiplied together---are more accurate, but are extremely inefficient
and not practical for online scheduling.
%Practicality is
%a central concern as several statistical methods for predicting
%application interference have proven not to scale beyond two
%applications \cite{Kambadur2010}; \ie they are impractical to
%use when co-scheduling three or more applications.

% Our approach -- split feature selection and model building into two
We therefore combine linear and non-linear approaches to produce
accurate and practical predictions.  We call our approach ESP for
\textbf{E}stimating co-\textbf{S}cheduled \textbf{P}erformance.
\SYSTEMESP{}'s key insight is to split regularization modeling into two
parts: \emph{feature selection} and \emph{model building}.  \SYSTEMESP{}
uses linear techniques to perform feature selection, but uses
quadratic techniques for model building.  The result is a highly
accurate predictor that is still practical and can be integrated into
real application schedulers.

\SYSTEMESP{} assumes there is a known (possibly very large) set of
applications that may be run on the system and some offline
measurements have been taken of these individual applications.
Specifically, \SYSTEMESP{} measures low-level hardware features like
cache misses and instructions retired during a training phase.  At
runtime, applications from this set may be launched in any arbitrary
combination.  The goal is to efficiently predict the interference (\ie
slowdown) of co-scheduled applications.
We evaluate \SYSTEMESP{} by integrating it into both single and
multi-node schedulers running on Linux/x86 servers.  In the
single-node case, we construct a batch scheduler that orders
application execution to minimize the total completion time.  In the
multi-node case, we build a first-come-first-serve scheduler that
assigns applications to nodes as they arrive to minimize application
slowdown.  We compare \SYSTEMESP{}-based schedulers to prior scheduling
techniques that use contention-aware heuristics to avoid interference
\cite{resense,merkel2010resource,Merlin}.  We also compare \SYSTEMESP{}'s
accuracy to predictors built with a number of cutting-edge
regularization methods. We find:
\begin{itemize}
\item The single-node \SYSTEMESP{} schedules are, on average, 27\% faster
  than techniques based on heuristics.  Even with its runtime
  overhead, the \SYSTEMESP{} results are only 5\% worse (on average) than
  an oracle that has perfect knowledge of interference and no
  overhead. See \secref{sched_single_proc}.
\item The multi-node \SYSTEMESP{} schedules are 60\% faster than activity
  vector based schedules and only 5\% to 13\% worse than an oracle.
  See \secref{sched-multi-node}.
\item Critically, \SYSTEMESP{} produces better results as more
  applications are scheduled.  \SYSTEMESP{} produces quantifiable
  performance predictions while heuristic techniques simply produce a
  binary decision: co-schedule or not.  \SYSTEMESP{}'s quantifiable
  predictions allow schedulers to make optimal decisions even when
  interference cannot be avoided. In contrast, heuristic techniques do
  not quantify interference and thus cannot rank decisions.  As the
  number of applications increases, the chance of heuristics making a
  very poor choice in the face of unavoidable contention also
  increases.  See \secref{exp:multi-node-job}.
\item \SYSTEMESP{} is more accurate than existing linear regression
  techniques in most cases.  When considering two applications,
  \SYSTEMESP{} is similar to existing regularized linear regression
  techniques.  Considering more than two applications, however,
  \SYSTEMESP{} is uniformly more accurate than standard linear regression
  techniques.  See \secref{st_model}.  For reasons explained in
  \secref{est-intro}, existing regularized regression methods with
  interaction terms cannot be evaluated for accuracy because their
  models are too complex to be implemented in practice.
\end{itemize}

% Contributions:
This work makes the following contributions:
%\begin{itemize}
\begin{inparaenum}[1)]
\item \SYSTEMESP{}, a regularization method for predicting application
  interference,
\item Demonstration of \SYSTEMESP{} in both single and multi-node
  schedulers,
\item Comparison to existing heuristic techniques on real systems,
\item Comparison of \SYSTEMESP{}'s predictive accuracy to other
  regularization methods,
\item Open-source code release\footnote{The code is available at
    https://github.com/ESPAlg/ESP}.
%\end{itemize}
\end{inparaenum}
\SYSTEMESP{} helps scheduling at multiple levels: it determines which
applications should run together on a single node, it determines which
node a new application should be scheduled on, and it avoids
disastrous decisions that heuristic schedulers cannot.  Support for
data analytics has become an important research topic in computing
systems, and this work explores how data analytics can be used to
improve computing systems.


We therefore combine linear and non-linear approaches to produce
accurate and practical predictions.
\SYSTEMESP{}'s key insight is to split regularization modeling into two
parts: \emph{feature selection} and \emph{model building}.  \SYSTEMESP{}
uses linear techniques to perform feature selection, but uses
quadratic techniques for model building.  The result is a highly
accurate predictor that is still practical and can be integrated into
real application schedulers.

\SYSTEMESP{} assumes there is a known (possibly very large) set of
applications that may be run on the system and some offline
measurements have been taken of these individual applications.
Specifically, \SYSTEMESP{} measures low-level hardware features like
cache misses and instructions retired during a training phase.  At
runtime, applications from this set may be launched in any arbitrary
combination.  The goal is to efficiently predict the interference (\ie
slowdown) of co-scheduled applications.

\PUNT{
\section{Outline}
The rest of the work is organized as follows. \Secref{sec:example}
provides a motivational example to build intuition.
\Secref{sec:notation} presents notation.
\Secref{sec:problemFormulation} formalizes the energy minimization
problem as a linear program and discusses the application-specific
parameters of this problem.  \Secref{sec:HBN} elaborates our probabilistic
graphical model and describes \SYSTEMLEO{} in full detail.
\Secref{sec:experiment} presents empirical studies on \SYSTEMLEO{}.
Related work is discussed in \Secref{sec:related} and the work
concludes in \secref{conclusion}.
}
