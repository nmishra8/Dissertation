\label{sec:related}
We discuss related work on energy and power optimization. Offline optimization techniques have been proposed (\eg \cite{Yi2003,LeeBrooks2006,CPR,ChenJohn2011,petabricksStatic}, but
they are limited by reliance on a robust training phase.  If behavior
occurs online that was not represented in the training data, then
these approaches may produce suboptimal results.

Several approaches augment offline model building with online
measurement.  For example, many systems employ control theoretic
designs which couple offline model building with online feedback
control
\cite{Wu2004,TCST,Chen2011,PTRADE,Heartbeats2,ControlWare,Agilos,Rajkumar,Sojka,Raghavendra2008}.
Over a narrow range of applications the combination of offline
learning and control works well, as the offline models capture the
general behavior of the entire class of application and require
negligible online overhead.  This focused approach is extremely
effective for multimedia applications
\cite{grace2,flinn99,flinn2004,xtune,TCST} and web-servers
\cite{Horvarth,LuEtAl-2006a,SunDaiPan-2008a}.  The goal of \SYSTEMLEO{},
however, is to build a more general framework applicable to a broad
range of applications.  \SYSTEMLEO{}'s approach is complementary to
control based approaches.  For example, incorporating \SYSTEMLEO{} into
control-based approaches might extend them to other domains even when
the application characteristics are not known ahead of time.

Some approaches have combined offline predictive models with online
adaptation
\cite{Zhang2012,packandcap,Winter2010,dubach2010,Koala,Cinder, wu2012inferred}.  For
example, Dubach et al.  propose such a combo for optimizing the
microarchitecture of a single core \cite{dubach2010}.  Such predictive
models have also been employed at the OS level to manage system energy
consumption \cite{Koala,Cinder}. \cite{wu2012inferred}.

Other approaches adopt an almost completely online model, optimizing
based only on dynamic runtime feedback
\cite{Li2006,Flicker,ParallelismDial,Ponamarev,petabricksDynamic,LeeBrooks}.
For example, Flicker is a configurable architecture and optimization
framework that uses only online models to maximize performance under a
power limitation \cite{Flicker}.  Another example, ParallelismDial,
uses online adaptation to tailor parallelism to application workload.


Perhaps the most similar approaches to \SYSTEMLEO{} are others that
combine offline modeling with online model updates
\cite{ICSE2014,Bitirgen2008,Ipek}.  For example, Bitirgen et al use an
artificial neural network to allocate resources to multiple
applications in a multicore \cite{Bitirgen2008}.  The neural network
is trained offline and then adapted online using measured feedback.
This approach optimizes performance but does not consider power or
energy minimization.

Like these approaches, \SYSTEMLEO{} combines offline model building and
with online model updates.  \emph{Unlike prior approaches, \SYSTEMLEO{}
  learns not a single best state, but rather all Pareto-optimal
  tradeoffs in the power/performance space (like those illustrated in
  \figref{fig:pareto})}.  These tradeoffs can be used to maximize
performance or to minimize energy across an application's entire range
of possible utilization.  There is a cost for this added benefit:
\SYSTEMLEO{}'s online phase is likely higher overhead than these prior
approaches that focus only on maximizing performance.  In that sense,
however, these approaches complement each other.  If fastest
performance is the goal, then prior approaches are likely the best
option.  If the goal is to minimize energy for a range of possible
performance, then \SYSTEMLEO{} produces near optimal energy.


We discuss related work in managing resources to meet performance
goals and reduce energy.

\section{Machine Learning}
We break learning for resource management into 3 categories: offline,
online, and hybrid approaches.

\noindent \textbf{Offline Learning} These approaches build models
before deployment and then use those fixed models to allocate
resources
\cite{Yi2003,LeeBrooks2006,CPR,ChenJohn2011,petabricksStatic}.  The
model-building phase is expensive, requiring both a large number of
samples and substantial computation.  Applying the model online,
however, is low overhead.  The main drawback is that the models are
not updated as the system runs: a problem for adapting workloads.
\PUNT{A good example of an offline approach applies learning to render
  web pages on mobile systems with low energy \cite{reddiHPCA2013}. It
  builds an offline model mapping web page features into performance
  for different core types.  When a new page is downloaded, the system
  estimates the resources needed to render the web page and uses the
  lowest energy configuration that meets user satisfaction.  This
  approach handles the complexity of allocating resources to webpage
  rendering, but cannot address dynamics; \eg{} when other apps run
  concurrently with the web browser.}  Carat is a good example of an
offline learner that aggregates data across multiple devices to
generate a report for human users about how to configure their device
to increase battery life \cite{carat}.  While both Carat and \SYSTEM{}
learn across devices, they have very different goals.  Carat returns
very high-level information to human users; \eg{} update a driver to
extend battery life.  \SYSTEM{} automatically builds and applies
low-level models to save energy.

\noindent \textbf{Online Learning} Online techniques observe the
current application to tune system resource usage for that application
\cite{Li2006,Flicker,ParallelismDial,Ponamarev,petabricksDynamic,LeeBrooks}.
For example, Flicker is a configurable architecture and optimization
framework that uses online models to maximize performance under a
power limitation \cite{Flicker}.  Another example, ParallelismDial,
uses online adaptation to tailor parallelism to application workload
\cite{ParallelismDial}.



\noindent \textbf{Hybrid Approaches} Some approaches combine offline
predictive models with online adaptation
\cite{Zhang2012,packandcap,Winter2010,dubach2010,Koala,Cinder,
  wu2012inferred}.  For example, Dubach et al.  use hybrid models to
optimize the microarchitecture of a single core \cite{dubach2010}.
Such predictive models have also been employed at the operating system
level to manage system energy consumption
\cite{Koala,Cinder,wu2012inferred}.  Other approaches combine offline
modeling with online updates \cite{JouleGuard,Bitirgen2008,Ipek}.  For
example, Bitirgen et al use an artificial neural network to allocate
resources to multiple applications in a multicore \cite{Bitirgen2008}.
The neural network is trained offline and then adapted online using
measured feedback.  This approach maximizes performance but does not
consider power or energy minimization.

\section{Control}
Almost all control solutions can be thought of as a combination of
offline model building with online adaptation.  The model building
involves substantial empirical measurement that is used to synthesize
a control system
\cite{Wu2004,TCST,Chen2011,PTRADE,POET,ControlWare,Agilos,Rajkumar,Sojka,Raghavendra2008}.
The combination of offline learning and control works well over a
narrow range of applications, as the offline models capture the
general behavior of a class of application and require negligible
online overhead.  This focused approach is extremely effective for
multimedia applications \cite{grace2,flinn99,flinn2004,xtune,TCST} and
web-servers \cite{Horvarth,LuEtAl-2006a,SunDaiPan-2008a} because the
workloads can be characterized ahead of time so that the models
produce sound control.

Indeed, the need for good models is the central tension in developing
control for computing systems.  It is always possible to build a
controller for a specific application and system by extensively
modeling that pair.  More general controllers, which work with a range
of applications, have addressed the need for models in various ways.
Some provide libraries that encapsulate control functionality and
require user-specified models
\cite{ControlWare,Sojka,Rajkumar,POET,SWiFT}.  Others automatically
synthesize both a model and a controller for either hardware
\cite{josep-isca2016} or software \cite{ICSE2014,FSE2015}.  JouleGuard
combines learning for energy efficiency with control for managing
application parameters \cite{JouleGuard}.  In JouleGuard, a learner
adapts the controller's coefficients to model uncertainty, but
JouleGuard's learner does not produce a new model for the controller.
Because JouleGuard's learner runs on the same device as the controlled
application, it must be computationally efficient and thus it cannot
identify correlations across applications or even different resource
configurations.  \SYSTEM{} is unique in that a separate learner
generates an application-specific model automatically.  By offloading
the learning task, \SYSTEM{} (1) combines data from many applications
and systems and (2) applies computationally expensive, but highly
accurate learning techniques.


\section{Application interference}
Accurate performance estimates are essential for solving scheduling
and resource allocation problems \cite{chiang2002impact}.  Accurate
estimates are difficult to obtain due to the complexity and diversity
of large-scale systems \cite{kanev2015profiling}.  A particular
challenge is modeling performance loss due to contention among
applications \cite{kambadur2012measuring}.  Better contention models
could improve system utilization while ensuring quality-of-service in
latency sensitive applications \cite{Bubble-flux}.

Several approaches investigate statistical and machine learning
techniques for estimating power, performance, and energy of a single
application running on a single system.  Many of these approaches
improve the time of developing new hardware designs, but are not
suitable for online resource management
\cite{Yi2003,LeeBrooks2006,CPR}.  Other approaches can predict the
power and performance of various resource allocations to single
applications \cite{Koala} or optimize energy efficiency under latency
constraints \cite{LEO}.  A recent approach combines machine learning
with control theory to guarantee energy consumption, but only for a
single application \cite{JouleGuard}.  Perhaps most similar to
\SYSTEMESP{} is the Mantis project which also uses higher-order
regularized regression models (based on Lasso) to predict smartphone
app performance \cite{kwon2013mantis}.  Mantis, however, does not
predict contention among multiple applications, which is \SYSTEMESP{}'s
focus.

Other approaches predict and mitigate contention in single-node
systems.  Many decide to co-schedule or not, but they do not produce
quantitative slowdown estimates.  For example, Dwyer et al. propose a
classifier that predicts whether contention will be high or low, but
this approach does not produce a numerical estimate
\cite{dwyer2012practical}.  Similarly, ReSense detects highly
contended resources and reacts to reduce that contention, but it never
estimates contention \cite{resense}. Another approach estimates
throughput (total system performance), but does not produce estimates
of individual application performance
\cite{xu2010cache,chen2010performance}.  Subramanian et al. propose an
approach that does produce accurate estimates of performance (within
9.9\%) based on only last-level cache access rate and memory bandwidth
\cite{subramanian2015slowdown}.  These results are achieved on a
simulator rather than a real system, however, and on our real system
these two features are not sufficient to predict contention with any
level of accuracy.  Another single-node system, D-Factor, uses
non-linear models to predict the slowdown of a new application given
current resource usage \cite{Lim2012dfactor}.  Unlike \SYSTEMESP{},
D-Factor is not capable of predicting how two applications will
interfere with each other if neither is currently running.

Several approaches estimate and mitigate contention to schedule for
multi-node systems.  The activity vectors approach works on single or
multi-node systems by maximizing the variance among resource usage in
co-scheduled applications \cite{merkel2010resource}.  This heuristic
makes intuitive sense -- applications with very different resource
needs are less likely to interfere with each other -- but this
approach does not produce quantitative estimates and therefore can
make bad decisions when contention is unavoidable.  Merlin, is
somewhat similar, in that it tries to estimate contended resources and
migrate virtual machines to areas of lower contention, but it also
does not produce slowdown estimates \cite{Merlin}.  DejaVu is a
machine learning approach that classifies application workloads and
then schedules according to known good schedules for the
classification \cite{dejavu}. DejaVu creates an \emph{interference
  index} which can be used to rank slowdowns and migrate VMs or
reallocate resources, but it does not produce accurate estimates of
the actual slowdowns incurred.  Similarly, Quasar \cite{quasar} and
Stay-Away \cite{stay-away} use classification schemes to predict and
mitigate interference, but neither produces performance estimates in
the face of interference.  Bubble-flux \cite{Bubble-flux}, an
improvement over the earlier Bubble-up \cite{Bubble-up} does produce
slowdown estimates and, like \SYSTEMESP{}, it is efficient enough to
consider interference among more than two applications.  The main
difference between Bubble-flux and \SYSTEMESP{} is that Bubble-flux uses
no offline prior information and must dynamically probe the system.
This lack of prior information means that Bubble-flux must suffer
either poor utilization or inaccurate schedules during the probing
phase.  \SYSTEMESP{} uses a highly accurate offline model and combines
that with online updates to its predictions to further improve
accuracy while making use of the vast amount of data available to
inform offline model building.

%\TODO{One paragraph on stats related work.}
We model interference estimation as a \emph{high-dimensional}
regression problem with prohibitively many dimensions.  Many
statistical methods address high-dimensionality (\eg SURE
\cite{fan2008sure}). In computer system performance, however,
measurable features are highly correlated and existing methods do not
provide high accuracy.  Other statistical models emit more accurate
predictors given correlated features (\eg \cite{yuan2006model} and
\cite{bien2013lasso}), but they do not scale to our problem size.
Even though we make a strong assumption that the interaction terms are
present only if their individual linear terms are significant, the
heuristic has very high accuracy and produces good schedules in
practice.
