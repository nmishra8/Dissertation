\section{Scheduling with \SYSTEMESP{}}
\label{section:scheduling}
We examine two use cases for \SYSTEMESP{}: (1) batch scheduling
applications on a single processor, (2) and scheduling dynamically
arriving applications on multiple processors.

\subsection{Single-node Scheduling}
\label{sec:single_proc}
%In the previous \secref{st_model}, we described an efficient algorithm
%to estimate the performance of applications when they run in a group.
We assume a set $S$ of applications with $|S| = m$. Each of the
applications have work $w_1, w_2,. . .  w_m$. They can be scheduled to
run alone or with other applications.  Our goal is to compute the
schedule that completes all applications' work in the minimal total
time.  The optimal schedule can be described as a solution to a linear
program if the performance of every application was known ahead of
time. Since we do not know the exact performance, our algorithm uses
\SYSTEMESP{} to predict the performance and then generate near-optimal
schedules.

We assume that at most $k$-tuples of applications can run together at
a time, meaning we must predict the performance for $2\binom{m}{2} +
3\binom{m}{3} + \ldots + k\binom{m}{k}$ application combinations.  The
intuition behind the restriction on $k$ is that the system will become
saturated at some point and it is no longer beneficial to continue to
add applications to a saturated node.
% We back this intuition by empirical findings in
% \secref{sched_single_proc} where we show that scheduling more than 4
% applications together does not give extra saving in terms of
% scheduling time.  Moreover in our case on a 32 core machine, we show
% that we can easily cut down the search space by restricting the
% applications to run at most 4 at a time and running more
% applications at a time does not offer extra saving in the scheduling
% time.

We first develop a linear program for optimal scheduling assuming
known performance. We will relax that assumption momentarily.  We also
assume that pre-emption is allowed. Then, an optimal schedule is given
by:
\begin{equation}
\begin{aligned}
\centering
			\label{eq:schedule_general}
			&   \y =\underset{  \y \geq 0}{\argmin} \;\;   \| \y \|_1\\
			&   \text{subject to} \;\;\;\; \mathbf{A} \y = \w
\end{aligned}
\end{equation}
This equation is quite simple, but the complexity is in the structure
of the $\y$ vector and $\mathbf{A}$ matrix.  Each element of $\y$ is
the time for a specific set of applications to be co-scheduled, while
the columns of $\mathbf{A}$ represent the each application's
performance when co-scheduled in that set.  For example, if $m = 3$,
the superset of all possible sets of applications is $\{ \{ 1 \},
\{2\}, \{3\}, \{1,2\}, \{1,3\},\{1,3\}, \{1,2,3\} \}$.  If we order
these sets, then $\y_j$ in Equation \eqref{eq:schedule_general} is the
time spent when all the applications in set $j$ run together and
$\mathbf{A}_{ij}$ is the speed of application $i$ when co-scheduled
with applications from set $j$.  This linear program has a sparse
solution with at most $2m$ non-zero solutions, hence the context
switching cost is not very high and is bounded by the total number of
applications.


Equation \eqref{eq:schedule_general} produces a minimal time schedule
but is not practical.  It requires that all application interference
is known \emph{a priori}.  Further, it assumes deterministic
application performance, but in real systems, performance is
stochastic and subject to inherent noise.  We therefore extend
Equation \eqref{eq:schedule_general} by assuming that the observed
performance $\hat{\mathbf{A}}$ is drawn from a Gaussian distribution:
\begin{equation}
\label{eq:opt_sched_gaussian}
\hat{A}_{ij} \sim \mathcal{N}(A_{ij},\Sigma_{ij}).
\end{equation}
So, instead of using $A_{ij}$, we sample $\hat{A}_{ij}$ to predict
$A_{ij}$ and $\Sigma_{ij}$.  Suppose $\tilde{\mathbf{A}}$ is our
prediction for $\mathbf{A}$, then instead of solving Equation
\eqref{eq:schedule_general} we solve:
\begin{equation}
\begin{aligned}
\centering
			\label{eq:schedule_general2}
			&   \y =\underset{  \y \geq 0}{\argmin} \;\;   \| \y \|_1\\
                        &  \text{subject to} \;\;\;\; \tilde{\mathbf{A}} \y = \w
\end{aligned}
\end{equation}
%where $\tilde{\mathbf{A}}$ represents an estimate for $\mathbf{A}$.
This equation is a proxy for Equation \eqref{eq:opt_sched_gaussian},
but it cannot guarantee that the required work is finished because it
uses predictions rather than true performance.  To deal with this
uncertainty we design an iterative algorithm which repeatedly solves
the approximate linear program in Equation
\eqref{eq:schedule_general2} until all work is finished.  This
approach accounts for inherent noise due to error in both performance
measurement and prediction.

\begin{algorithm}[!t]
\caption{Iterative Scheduling Algorithm}
\begin{algorithmic}[1]
\small
\REQUIRE Tolerance: $\text{TOL} = 10^{-6}$ \\
Performance prediction matrix from \SYSTEMESP{}: $\tilde{\mathbf{A}}$ \\
Total work: $\w$ \\
Exploration-exploitation parameter: $\eta$ ($\eta \geq 1$)\\
\STATE Initialize scheduling matrix $\y^{(0)} = 0$, work remaining: $ \w_{rem} = \w$, total scheduling time: $s = 0$.
\WHILE{$\|\w_{rem}\|_2 \geq \text{TOL}$}
	\STATE Work to be processed, $ \w_{sent} = \w_{rem}/\eta$.
    \STATE Solve linear program based on the prediction, \\ $\y = \argmin_{\y \in \mathcal{C}} \|\y\|_1 $, \\where $ \mathcal{C} = \{ \y: \tilde{\mathbf{A}} \y = \w_{sent} ,\y \geq 0 \}$.
	\STATE Schedule/run applications according to time slices given by $\y$. Update scheduling time as, $s = s + \|A \y\|_1$.
	\STATE Update predicted performance $\mathbf{\tilde{A}}$ based on true performance observed $\mathbf{\hat{A}}$ and update remaining work,  $\w_{rem} = ( \w_{sent} -  \mathbf{\hat{A}} \y )_{+}$.
\ENDWHILE
\RETURN Total scheduling time: $s$.
\end{algorithmic}
\label{alg:ItrSched}
\end{algorithm}

%In this algorithm $\hat{A}$ is a random variable representing stochastic performance and its value is different at each iteration.
Algorithm \ref{alg:ItrSched} takes four parameters: an error tolerance
(defaulted to $10^{-6}$), \SYSTEM's predicted performance matrix, a
vector representing the total work for each application, and a scalar
$\eta$ controlling the trade-off between exploration-exploitation. If
we believe that \SYSTEMESP{}'s performance prediction is very good, we
can set $\eta = 1$ to run fewer iterations, saving computation. If we
have less confidence, then we could send a smaller amount of work to
the linear program thereby reducing the error at each iteration and
learning more about the particular application mix scheduled that
iteration.

This algorithm loops until all work is complete.  For each loop
iteration, the algorithm takes a \emph{step} to complete some work,
with the step size inversely proportional to $\eta$.  It then solves
Equation \eqref{eq:schedule_general2} using the predicted performance
and schedules the applications according to this solution.  After
running the schedule for the specified time, the algorithm updates the
performance prediction using its latest observations and updates the
work vector with the work accomplished in this step.  Given perfect
knowledge of the application interference and no system noise, the
algorithm requires only a single step and would be optimal.

\subsection{Multi-node Scheduling}
\label{sec:schedule_multi}
We now consider scheduling dynamically arriving applications on
multiple processors. Applications arrive in a stream and a centralized
scheduler assigns an application to a processor (which may already
have applications running). We assign jobs in a first-come-first-serve
order; a new job is immediately assigned a processor with the goal of
minimizing job completion time.

The multi-node scheduler (see Algorithm \ref{alg:ItrSched-multi})
takes as input: (1) an error tolerance (again defaulted to $10^{-6}$),
(2) \SYSTEMESP{}'s performance prediction $\tilde{A}$, (3) a job sequence
(we do not look ahead, so in practice the sequence does not need to be
known in advance), (4) the number of nodes $q$, and the
exploration-exploitation trade-off $\eta$.  Each job is denoted as
$J_i = (a_i,v_i)$, where $a_i$ is the application index and $v_i$ is
that application's work.  The algorithm loops until all jobs are
completed.  Each iteration takes the next job and determines the
expected work and time if assigned to each processor (lines 4--7).
It chooses the processor that has the fastest predicted completion
time (line 8) and schedules that job on the processor using Algorithm
\ref{alg:ItrSched} (line 9).

\begin{algorithm}[!t]
\caption{Multi-node Iterative Scheduling Algorithm}
\begin{algorithmic}[1]
\small
\REQUIRE Tolerance: $\text{TOL} = 10^{-6}$\\
Performance prediction matrix from \SYSTEMESP{}: $\tilde{\mathbf{A}}$ \\
Jobs arriving in stream: $J_1, J_2, \ldots, J_T $ \\
Number of processors: $q$. \\
Exploration-exploitation parameter: $\eta$ ($\eta \geq 1$) \\
\STATE Initialize work matrix: $\mathbf{W} = \mathbf{0}_{m \times q}$
\FORALL{i = 1:T}
\STATE Obtain job $J_i = (a_i, v_i)$. where $a_i$ is the application index and $v_i$ denotes the work for that application.
\FORALL{j = 1:q}
\STATE Expected work, $\w_{tmp} = \mathbf{W}(:,j)$; updated with new job's work $w_{tmp}[i] = w[a_i,j] + v_i$
\STATE Expected schedule time, $s[j] = \min_{\y \in \mathcal{C}} \|\y\|_1 $, \\where $ \mathcal{C} = \{ \y: \tilde{\mathbf{A}} \y = \w_{tmp} ,\y \geq 0 \}$.
\ENDFOR
\STATE Greedily choose processor $P: P = \argmin_{i\in [q]} s_i$ with least expected scheduling time.
\STATE Run Algorithm \ref{alg:ItrSched} for processor $P$ with $\w_{tmp}$ amount of Total work, $\tilde{\mathbf{A}}$ as the performance prediction matrix and $\eta = 5$.
\ENDFOR
\RETURN Total scheduling time for all processors: $\mathbf{s}$.
\end{algorithmic}
\label{alg:ItrSched-multi}
\end{algorithm}
