\section{Energy Minimization}
\label{sec:problemFormulation}

% Example on x264
This section formalizes the problem of minimizing an application's
energy consumption for some \emph{performance constraint}; \ie work
that should be accomplished by a particular deadline.  We assume a
configurable system where each configuration has different
application-specific performance and power characteristics. Our aim is
to select the configuration that finishes the work by the deadline
while minimizing the energy consumption.

Formally, the application must accomplish $W$ work units in time $T$.
The system has a set of configurations (\eg combinations of cores and
clockspeeds) denoted by $\mathcal{C}$. Assuming that each
configuration $c \in \mathcal{C}$ has an application-specific
performance (or work rate) $r_c$ and power consumption $p_c$, then we
formulate the energy minimization problem as a linear program in
Equation \eqref{eq:controller}:
\begin{equation}
\begin{aligned}
&   \underset{\mathbf{t} \geq \mathbf{0}}{\text{min}}  
&&   \sum_{c \in \mathcal{C}} p_c t_c, \\
&   \text{subject to} &&  \sum_{c \in \mathcal{C}} r_c t_c = W, \\
&&&	 \sum_{c \in \mathcal{C}} t_c \leq T.
\end{aligned}
\label{eq:controller}
\end{equation}
where $p_c$: Power consumed when running on $c^{th}$ configuration;
$r_c$: Performance rate when running on $c^{th}$ configuration; $W$:
Work that needs to be done by the application; $t_c$: Time spent by
the application in $c^{th}$ configuration; $T$: Total run time of the
application. The linear program above finds the times $t_c$ during
which the application runs in the $c^{th}$ configuration so as to
minimize the total energy consumption and ensure all work is completed
by the deadline.  The values $p_c$ and $r_c$ are the key to solving
this problem.  If they are known, the structure of this linear program
allows the minimal energy schedule to be found using convex
optimization techniques \cite{LinearProgramming}. 

This formulation is abstract so that it can be applied to many
applications and systems.  To help build intuition, we relate it to
our \texttt{Kmeans} example.  For \texttt{Kmeans} the workload is the
number of samples to cluster.  The deadline $T$ is the time by which
the clustering must be completed.  Configurations represent assigning
\texttt{Kmeans} different resources.  In \secref{example}, we
restricted configurations to be assignment of cores.  In
\secref{experiment}, we will expand configurations to include
assignment of cores, clockspeed, memory controllers, and hyperthreads.
For \texttt{Kmeans}, each assignment of resources results in a
different rate of computation (points clustered per time) and power
consumption.  

Unfortunately, power and performance are entirely application
dependent.  For many applications, these values also vary with varying
inputs.  Hence, for any new application in use we do not know the
values of these coefficients. One way to solve the problem would be
run this new application on each configuration in a brute force
manner. But, as we pointed out earlier, we might have very large
number of configurations and the brute force approach may not be
tractable. Alternatively, we can just run the application in small
subset of configurations and use these measurements to estimate the
behavior of unmeasured configurations.  We might also consider using
the data from other applications from the same system to estimate
these parameters (we can collect this data offline). The question now
is how do we utilize this data to find our estimates. One simple yet
clever thing would be to simply take a mean of $p_c$ (similarly for
$r_c$) across all the applications. Now, this offline method will work
well for any application that follows the general trend exhibited by
all prior applications. Another quick solution could be to just use
the small subset of collected sample and run a multivariate polynomial
regression on the configuration parameters vs $p_c$ (or $r_c$) to
predict power (performance) in all other configurations.  This online
method might not work for all the applications s because they might
have local minima or maxima that are not captured by a small sample.
In the next section, Section \ref{sec:HBN} we give details of
\SYSTEM{}, our solution to this problem which uses both the data from
the current application and previously seen applications for fast,
accurate estimates.
%\section{Notation}
%Throughout this 