\section{Summary}
% Contributions:

This work explores machine learning methods for predicting
application interference in computing systems.  Specifically, we
explore several state-of-the-art regularization techniques for
high-dimensional problems---when many more features are available than
samples---and we conclude that existing linear techniques are not
accurate enough, while existing higher-order techniques are accurate
but slow.  Inspired by these observations we present \SYSTEMESP{}, a
combination of linear feature selection with higher-order model
building that achieves the practicality of linear models with the
accuracy of higher-order models.  We have demonstrated that
\SYSTEMESP{}'s quantitative predictions produce significantly better
schedules than existing heuristics for both single and multi-node
systems, with up to 1.8$\times$ improvement in application completion
time and significantly lower variance.  Additionally, \SYSTEMESP{}
achieves much higher prediction accuracies than prior
approaches---over 93\% when considering three or more applications.
We have made the source code available so that others may improve on
or compare with \SYSTEMESP{}.
This work makes the following contributions:
%\begin{itemize}
\begin{inparaenum}[1)]
\item \SYSTEMESP{}, a regularization method for predicting application
  interference,
\item Demonstration of \SYSTEMESP{} in both single and multi-node
  schedulers,
\item Comparison to existing heuristic techniques on real systems,
\item Comparison of \SYSTEMESP{}'s predictive accuracy to other
  regularization methods,
\item Open-source code release\footnote{The code is available at
    https://github.com/ESPAlg/ESP}.
%\end{itemize}
\end{inparaenum}
\SYSTEMESP{} helps scheduling at multiple levels: it determines which
applications should run together on a single node, it determines which
node a new application should be scheduled on, and it avoids
disastrous decisions that heuristic schedulers cannot.  Support for
data analytics has become an important research topic in computing
systems, and this work explores how data analytics can be used to
improve computing systems.


We therefore combine linear and non-linear approaches to produce
accurate and practical predictions.
\SYSTEMESP{}'s key insight is to split regularization modeling into two
parts: \emph{feature selection} and \emph{model building}.  \SYSTEMESP{}
uses linear techniques to perform feature selection, but uses
quadratic techniques for model building.  The result is a highly
accurate predictor that is still practical and can be integrated into
real application schedulers.

\SYSTEMESP{} assumes there is a known (possibly very large) set of
applications that may be run on the system and some offline
measurements have been taken of these individual applications.
Specifically, \SYSTEMESP{} measures low-level hardware features like
cache misses and instructions retired during a training phase.  At
runtime, applications from this set may be launched in any arbitrary
combination.  The goal is to efficiently predict the interference (\ie
slowdown) of co-scheduled applications.
