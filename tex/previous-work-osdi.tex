\section{Related work}
Accurate performance estimates are essential for solving scheduling
and resource allocation problems \cite{chiang2002impact}.  Accurate
estimates are difficult to obtain due to the complexity and diversity
of large-scale systems \cite{kanev2015profiling}.  A particular
challenge is modeling performance loss due to contention among
applications \cite{kambadur2012measuring}.  Better contention models
could improve system utilization while ensuring quality-of-service in
latency sensitive applications \cite{Bubble-flux}.

Several approaches investigate statistical and machine learning
techniques for estimating power, performance, and energy of a single
application running on a single system.  Many of these approaches
improve the time of developing new hardware designs, but are not
suitable for online resource management
\cite{Yi2003,LeeBrooks2006,CPR}.  Other approaches can predict the
power and performance of various resource allocations to single
applications \cite{Koala} or optimize energy efficiency under latency
constraints \cite{LEO}.  A recent approach combines machine learning
with control theory to guarantee energy consumption, but only for a
single application \cite{JouleGuard}.  Perhaps most similar to
\SYSTEM{} is the Mantis project which also uses higher-order
regularized regression models (based on Lasso) to predict smartphone
app performance \cite{kwon2013mantis}.  Mantis, however, does not
predict contention among multiple applications, which is \SYSTEM{}'s
focus.

Other approaches predict and mitigate contention in single-node
systems.  Many decide to co-schedule or not, but they do not produce
quantitative slowdown estimates.  For example, Dwyer et al. propose a
classifier that predicts whether contention will be high or low, but
this approach does not produce a numerical estimate
\cite{dwyer2012practical}.  Similarly, ReSense detects highly
contended resources and reacts to reduce that contention, but it never
estimates contention \cite{resense}. Another approach estimates
throughput (total system performance), but does not produce estimates
of individual application performance
\cite{xu2010cache,chen2010performance}.  Subramanian et al. propose an
approach that does produce accurate estimates of performance (within
9.9\%) based on only last-level cache access rate and memory bandwidth
\cite{subramanian2015slowdown}.  These results are achieved on a
simulator rather than a real system, however, and on our real system
these two features are not sufficient to predict contention with any
level of accuracy.  Another single-node system, D-Factor, uses
non-linear models to predict the slowdown of a new application given
current resource usage \cite{Lim2012dfactor}.  Unlike \SYSTEM{},
D-Factor is not capable of predicting how two applications will
interfere with each other if neither is currently running.

Several approaches estimate and mitigate contention to schedule for
multi-node systems.  The activity vectors approach works on single or
multi-node systems by maximizing the variance among resource usage in
co-scheduled applications \cite{merkel2010resource}.  This heuristic
makes intuitive sense -- applications with very different resource
needs are less likely to interfere with each other -- but this
approach does not produce quantitative estimates and therefore can
make bad decisions when contention is unavoidable.  Merlin, is
somewhat similar, in that it tries to estimate contended resources and
migrate virtual machines to areas of lower contention, but it also
does not produce slowdown estimates \cite{Merlin}.  DejaVu is a
machine learning approach that classifies application workloads and
then schedules according to known good schedules for the
classification \cite{dejavu}. DejaVu creates an \emph{interference
  index} which can be used to rank slowdowns and migrate VMs or
reallocate resources, but it does not produce accurate estimates of
the actual slowdowns incurred.  Similarly, Quasar \cite{quasar} and
Stay-Away \cite{stay-away} use classification schemes to predict and
mitigate interference, but neither produces performance estimates in
the face of interference.  Bubble-flux \cite{Bubble-flux}, an
improvement over the earlier Bubble-up \cite{Bubble-up} does produce
slowdown estimates and, like \SYSTEM{}, it is efficient enough to
consider interference among more than two applications.  The main
difference between Bubble-flux and \SYSTEM{} is that Bubble-flux uses
no offline prior information and must dynamically probe the system.
This lack of prior information means that Bubble-flux must suffer
either poor utilization or inaccurate schedules during the probing
phase.  \SYSTEM{} uses a highly accurate offline model and combines
that with online updates to its predictions to further improve
accuracy while making use of the vast amount of data available to
inform offline model building.

%\TODO{One paragraph on stats related work.}
We model interference estimation as a \emph{high-dimensional}
regression problem with prohibitively many dimensions.  Many
statistical methods address high-dimensionality (\eg SURE
\cite{fan2008sure}). In computer system performance, however,
measurable features are highly correlated and existing methods do not
provide high accuracy.  Other statistical models emit more accurate
predictors given correlated features (\eg \cite{yuan2006model} and
\cite{bien2013lasso}), but they do not scale to our problem size.
Even though we make a strong assumption that the interaction terms are
present only if their individual linear terms are significant, the
heuristic has very high accuracy and produces good schedules in
practice.
