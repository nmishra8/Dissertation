\section{Motivation}
\label{sec:regression}
This section discusses current state-of-the-art regularization methods
and how they could be used to predict application interference.  A
complete coverage of these techniques is beyond the scope of this
work---our goal is to present enough background for readers to
understand \SYSTEMESP{}'s unique approach to interference prediction.

We use a running example to build intuition.  This example is
simplified, but our hope is that it provides sufficient intuition for
readers to understand the full results.

\noindent \textbf{Example~~}\textit{Suppose we have 4 applications
  given as, \{\texttt{bfs, cfd, jacobi, kmeans} \}.  For each
  application, we measure 4 features of its execution when run by
  itself: instructions per clock \texttt{(IPC)}, L2 Access
  Rate\texttt{(L2)}, L3 Access Rate\texttt{(L3)} and its memory
  bandwidth \texttt{(MEM)}.  The goal is to predict the slowdown that
  each of these applications will experience when run together in any
  combination knowing only these 4 features for each application run
  individually.  }

\subsection{Regularization Overview}
We are interested in predicting applications' slowdown when
co-scheduled with other applications as a function of individual
applications' features. The following is a general formulation of this
problem:
\begin{equation}
\label{eq:stat_model_general}
\z \sim f(\mathbf{X}),
\end{equation}
where $\z$ is a vector each element is one application's slowdown when
run with the other applications.  $\mathbf{X}$ is a matrix containing
the measured features of applications when run individually.  $f(.)$
is the\emph{prediction function} that maps the measured features into
predicted slowdown.  The $\sim$ sign indicates that $\z$ is a random
variable drawn from a distribution given by $f(.)$.
% We pin down this formulation more explicitly later in Equation
% \eqref{eq:system_general_equation} in \secref{system_subsection}.

To go from Equation \eqref{eq:stat_model_general} to a useful model
two issues must be addressed: (1) which features we choose to include
in $\mathbf{X}$ and (2) what function $f(.)$ best represents this
problem. In fact, these issues are intertwined; depending on the
features we include we can choose different functions.  It is
important to choose a function that captures the data without
overfitting so that the model's generalization error is small.  The
model also must be computationally fast so that it is practical.

% In \secref{est-intro} we introduce the regularized regression models
% which form the basis for our algorithm. We discuss answers to the
% questions we posed above in detail in \secref{est:regularized_lin_reg}
% and in \secref{est:system} we describe our algorithm which is
% computationally fast and has very high prediction accuracy.\PUNT{ (See
%   \secref{st_model} for details on empirical results.) and low
%   over-head (\TODO{overhead analysis}). }

%{\setlength{\parindent}{0cm}{
\noindent \textbf{Example~~}\textit{ Suppose we want to predict the
  interference of \texttt{bfs} and \texttt{cfd} when co-scheduled.
  The vector $z$ represents slowdown, with the first element
  \texttt{bfs}'s slowdown and the second is \texttt{cfd}'s.  $X$ in
  Equation \eqref{eq:stat_model_general} is constructed using
  \texttt{IPC}, \texttt{L2}, \texttt{L3} and \texttt{MEM} of the
  \texttt{bfs} and \texttt{cfd} when they run in isolation. The first
  four columns of $X$ are the features for \texttt{bfs} and the last
  four columns are \texttt{cfd}'s features.  The goal is to find the
  $f(.)$ that produces the best prediction of slowdown.  }
%}

\subsection{Linear Regression}
\label{sec:est-intro}
Regression models are the most commonly used statistical models for
predictions.  A linear model predicts outcomes as a linear combination
of input features:
\begin{equation}
\label{eq:linear}
\z = \mathbf{X}\boldsymbol{\beta} + \boldsymbol{\epsilon},
\end{equation}
$\z\in \R^{n}$ is the dependent variable to be predicted.  $\mathbf{X}
\in \R^{n \times p}$ is the measured independent features.
$\boldsymbol{\boldsymbol{\beta}} \in \R^{p}$ is the coefficient vector
that combines the features to produce the prediction.
$\boldsymbol{\epsilon} \in \R^{n}$ is a vector representing inherent
noise.

Building a linear model is the process of determining
$\boldsymbol{\boldsymbol{\beta}}$ by running experiments and measuring
both the features and the corresponding outcomes.  These measurements
are samples of $\z$ and $\mathbf{X}$, written as $(s(\z),
s(\mathbf{X}))$.  After sampling, $s(\z)$ and $s(\mathbf{X})$ are
known and we can solve for $\boldsymbol{\beta}$. Then we can use
$\boldsymbol{\boldsymbol{\beta}}$ and new features ($\mathbf{X}$) to
predict $\z$ for instances that have not been measured.

\subsection{Regularized Linear Regression}
\label{sec:est:regularized_lin_reg}
When determining $\boldsymbol{\boldsymbol{\beta}}$, if the number of
samples in $\z$ (or length of $s(\z)$) is less than the number of
features $p$ the problem is ill-posed; \ie $\mathbf{X}$ is not
invertible and there are infinitely many solutions. This particular
case where $p>n$ is called the \textit{high-dimensional} setting.
Machine learning researchers have developed several
\emph{regularization} methods which add structure to high-dimensional
problems to make them solvable.
% These methods have strong theoretical backing that they reduce the
% generalization errors in the model and have shown the work well in
% practice.


\noindent \textbf{Example~~}\textit{ To apply linear regression,
  $f(.)$ is chosen to be $f(.) = \mathbf{X}\boldsymbol{\beta}$ and
  $\boldsymbol{\beta}$ is the parameter that we would like to learn
  from the data. For example, we could observe 3 pairs from the set of
  6 pairs (for 4 applications in our example) and predict the
  performance for the other 3 pairs. $\z$ denotes the performance
  vector for our applications, thus for 3 pairs, $\z \in \R^{6}$. The
  matrix $\mathbf{X}$ in Equation \eqref{eq:stat_model_general} is
  $\mathbf{X} \in \R^{6 \times 8}$ and $\boldsymbol{\beta} \in
  \R^{8}$.  Now, the system of equations $\z =
  \mathbf{X}\boldsymbol{\beta}$ has infinitely many solutions so we
  must \emph{regularize}---\ie add more structure to---the problem.
%}
}

There are several methods to add structure to a regression problem.
In general, none are uniformly better than the others and their
performance is data-dependent. Hence, standard practice is to use the
model with the highest out-of-sample predictive power; \ie to separate
samples into training and test data, build multiple models with the
training data, and use the model that most accurately predicts the
test data.

\subsubsection{Ridge regularization}
One way to add structure to the problem is to include additional
constraints.  \emph{Ridge} regression penalizes large coefficients in
$\boldsymbol{\beta}$ \cite{hoerl1988ridge} by requiring that
$\|\boldsymbol{\beta}\|^2_2 \leq t$, where $t$ is a threshold:
\begin{equation}
\label{eq:ridge}
\min_{\boldsymbol{\beta}} \|\z - \mathbf{X}\boldsymbol{\beta} \|^2_2 \;\; \text{  s.t.  } \| \boldsymbol{\beta} \|^2_2 \leq t
\end{equation}

\noindent \textbf{Example~~}\textit{ When we add the ridge regularizer
  to our example, the modified problem is $\|\z -
  \mathbf{X}\boldsymbol{\beta}\|^2_2$ and $\sum_{i=1}^8
  \boldsymbol{\beta}^2[i]<t$. This problem has a unique solution for a
  given $t$. This regularization shrinks large $\boldsymbol{\beta}$
  values but does not make them 0. Hence, even if a feature $i$ is not
  important it receives a positive coefficient $\beta_i$.  The
  downside of this approach, then, is that it may incorporate
  irrelevant features into the model.  In our example, it is unlikely
  that L2 Access Rate is a valuable feature for predicting application
  interference, because L2 caches are private in most server class
  processors and therefore not a source of contention.  }
%}

\subsubsection{Lasso regularization}
\emph{feature selection} is another regularization method equivalent
to setting some elements of $\boldsymbol{\beta}$ to $0$, so those
features cannot influence $\z$.  Formally, feature selection can be
achieved by adding $\mathcal{L}_0$
constraint---$\|\boldsymbol{\beta}\|_0 \leq t$---to Equation
\eqref{eq:linear}, but doing so makes the problem non-convex and
NP-hard.  $\mathcal{L}_1$ regularization---known as \emph{lasso}---is
the best convex relaxation for $\mathcal{L}_0$ regularization:
\begin{equation}
\label{eq:lasso}
\min_{\boldsymbol{\beta}} \|\z - \mathbf{X}\boldsymbol{\beta} \|^2_2 \;\; \text{  s.t.  } \|\boldsymbol{\beta} \|_1 \leq t
\end{equation}
Using lasso, the number of selected features will be smaller than the
number of samples. A potential drawback, however, is that it cannot
capture models where there are more relevant features than samples.

%{\setlength{\parindent}{0cm}{
\noindent \textbf{Example~~}\textit{ The modified example looks like,
  $\|\z - \mathbf{X}\boldsymbol{\beta}\|^2_2$ and $\sum_{i=1}^8
  |\boldsymbol{\beta}[i]|<t$. Lasso regularization will set $n-p$
  features to 0.  In our example, that means three features will be
  zeroed out. This could be a problem because it is likely that IPC,
  L3 accesses, and Memory Bandwidth for both applications (\ie a total
  of 6 features for the two applications) will be necessary to predict
  interference as all three of these features correspond to shared
  hardware structures.  Specifically, lasso regularization will
  probably zero-out the L2 Accesses for both applications, but by
  construction, it must also zero-out at least one of the relevant
  features, likely producing lower accuracy estimates.  }
%}

\subsubsection{Elastic-net regularization}
\label{sec:est:elastic-net}
\textit{Elastic-net} addresses the drawbacks of the two previous
techniques \cite{zou2005regularization}:
\begin{equation}
\label{eq:elastic-net}
\min_{\boldsymbol{\beta}} \|\z - \mathbf{X}\boldsymbol{\beta} \|^2_2 \;\; \text{  s.t.  }
\alpha \| \boldsymbol{\beta} \|^2_2 + (1-\alpha)\| \boldsymbol{\beta} \|_1 \leq t
\end{equation}
where $\alpha = {\lambda_1}/({\lambda_1+\lambda_2})$ and $\lambda_1$
and $\lambda_2$ are the regularization parameters. In practice,
setting $\alpha = 0.5$ is common and $\lambda_1$ is set during model
training using cross-validation. Thus under these settings elastic-net
is,
\begin{equation}
\label{eq:elastic-net2}
\text{EN}(\z, \mathbf{X}) = \argmin_{  \| \boldsymbol{\beta} \|^2_2 + \| \boldsymbol{\beta} \|_1 \leq t  } \|\z - \mathbf{X}\boldsymbol{\beta}\|^2_2
\end{equation}
Elastic-net \emph{groups} variables so that strongly correlated
variables tend to be selected or rejected together.

\noindent \textbf{Example~~}\textit{ Adding elastic-net to our
  example, the modified problem looks like, $\|\z -
  \mathbf{X}\boldsymbol{\beta}\|^2_2$ and $\sum_{i=1}^8 \alpha
  |\boldsymbol{\beta}[i]| + (1-\alpha)\|\boldsymbol{\beta}[i]\|^2<t$.
  The regularization in this case would shrink large values for
  $\boldsymbol{\beta}$ and some coefficients would shrink to 0. It
  would zero-out unimportant feature like \texttt{L2} for both
  applications, yet capture all the important features (IPC, L3, MEM)
  for both applications even when the number of samples is smaller
  than the number of features.  }
%}

\subsection{Higher-order Models}
In some cases, linear models do not accurately predict the problem.
One higher-order approach is to add \emph{interaction} terms to the
model, meaning that the dependent variable is possibly a
multiplicative combination of some features.

A model with interaction terms can be found by adding additional terms
to $\mathbf{X}$ and $\mathbf{\beta}$.  Thus, a high dimensional
problem becomes even higher dimensional, increasing model complexity
and overhead. For a matrix $\mathbf{X}$ with dimensions $n \times p$,
$n \ll p$ \textit{elastic-net} is $\mathcal{O}(p^3)$
\cite{zou2005regularization}, hence for a quadratic model the
computational complexity blows up to $\mathcal{O}(p^6)$.  Even though
this model is richer and captures complex interactions among features,
it is prohibitively expensive.  \SYSTEMESP{}'s motivation is to achieve
the prediction accuracy of interaction terms while maintaining the
practicality of linear models.

\noindent \textbf{Example~~}\textit{ We believe the interaction
  between our features captures the interference more accurately than
  a linear model.  However, we do not know which of these feature
  interactions are important in advance---\eg is it IPC and L3, L3 and
  MEM, the L3 features for both applications, or something else?
  Clearly, even in our simple example, the design space has become
  much more complex.  Specifically, we capture the new interaction
  terms by extending the feature matrix $\mathbf{X}$ to
  $\mathbf{\tilde{X}}$, which has 8 linear terms plus ${8 \choose 2}$
  higher order terms. The new design matrix $\mathbf{\tilde{X}}$ in
  Equation \eqref{eq:stat_model_general} is $\mathbf{\tilde{X}} \in
  \R^{6 \times 36}$ and $\boldsymbol{\beta} \in \R^{36}$. We
  see---even for our simple example---the model complexity has greatly
  increased.  }
%}


\subsection{ A New Regularization Method}
\label{sec:est:system}
To summarize, regression models map features into predictions.  When
the feature space is large, regularization adds structure to make the
problem well-formed.  Higher-order models may provide more accurate
predictions, but increase the cost of both training and applying the
model.

Inspired by these observations, we present \SYSTEMESP{}, which splits
regression modeling into two parts: (1) feature selection and (2)
model building with interactive terms.  First, \SYSTEMESP{} builds a
linear model with elastic-net, which greatly reduces feature size
without capturing interaction terms.  Second, \SYSTEMESP{} builds a
higher-order model with interaction terms using just those features
selecting in the first step.  By reducing the feature size in the
first step, the complexity of the higher-order model remains tractable
and we get the benefits of both approaches: \emph{highly accurate
  predictions with manageable complexity.}


Given a set of $m$ applications $k$ applications to be co-scheduled,
there are $n = {m \choose k}$ possible sets of $k$ applications. We
use $p$ to denote the number of features.  Let $f^{(k)}(.)$ be the
prediction function; \ie $f^{(k)}(.)$ predicts the slowdown of each of
the $k$ co-scheduled applications using features measured when each
application runs individually:
\begin{equation}
\label{eq:system_general_equation}
\z^{(k)} = f^{(k)}(\mathbf{X}^{(k)}) + \boldsymbol{\epsilon},
\end{equation}
where $ \z^{(k)} \in \R^{nk}$ is the slowdown vector,
$\mathbf{X}^{(k)} \in \R^{nk \times 2p}$ is feature matrix and
$\boldsymbol{\epsilon} \in \R^{nk}$ is the Gaussian error vector. For
any index $j = \text{index($i$,$S$)}$ of vector $\z$, $z_j$ is the
slowdown of application-$i$ when co-scheduled with applications from
set $S$. $X_{(j,:)}$ = [features of application-$i$, $\sum_{s \in
  S}$(features of application-$s$)]. In practice, the set of possible
features includes everything that can be measured with the Intel
performance counter monitor tool \cite{willhalm2012intel} on Linux x86
systems (see Table \ref{llf-table}). Taking these measurements for
each processor core we get $p=409$ unique features per individual
application.

\begin{table*}[htp!]
\centering
\footnotesize
\caption{Category of low level features using Intel Performance Counter Monitor. }
\begin{tabular}{ ll }
 \hline
 \hline
 \bf{Code} & \bf{Low level feature description} \\

 \hline
 \texttt{IPC}   & Instructions per CPU cycle    \\
 \texttt{AFREQ} & Relation to nominal CPU frequency while in active state    \\
 \texttt{L2MISS}& L2 cache misses (including other core's L2 cache *hits*)   \\
\texttt{L2HIT}  & L2 cache hit ratio (0.00-1.00)  \\
 \texttt{L2CLK} & Ratio of CPU cycles lost due to missing L2 cache but still hitting L3 cache (0.00-1.00)  \\
 \texttt{L3HIT} & L2 cache hit ratio (0.00-1.00)  \\
  \texttt{L3CLK}& Ratio of CPU cycles lost due to L3 cache misses (0.00-1.00)  \\
 \texttt{READ}  & Bytes read from memory controller (in GBytes) \\
 \texttt{WRITE} & Bytes written to memory controller (in GBytes)  \\
 \texttt{COR-RES}  &  Core residency  \\
\texttt{PKG-RES}  &  Cackage residency   \\
 \texttt{TEMP}  &  Temperature reading in 1 degree Celsius relative to the TjMax temperature (thermal headroom) \\
\texttt{EXEC}  &  Instructions per nominal CPU cycle \\
\texttt{FREQ}  &  Relation to nominal CPU frequency='unhalted clock ticks'/'invariant timer ticks' (includes Intel Turbo Boost) \\
\texttt{L3MISS}  &  L3 cache misses \\
\texttt{ENERGY}  &  Energy consumed  \\
 \hline
 \hline
\end{tabular}
\label{llf-table}
\end{table*}

\begin{algorithm}[!t]
\caption{\SYSTEM }
\begin{algorithmic}[1]
\REQUIRE  Training samples: $\left( s(\z^{(k)}), \; \x^{(k)} = s(\mathbf{X}^{(k)}) \right)$ ; Feature matrix: $\mathbf{X}^{(k)}$,
	\STATE Variable selection using elastic-net on linear model:  $\mathcal{S}: \{ i \in \mathcal{S} \; \text{if} \; \boldsymbol{\beta}^{(k)}[i] \neq 0\}$, $ \text{where} \; \boldsymbol{\beta}^{(k)} = \text{EN}(s(\z^{(k)}), \x^{(k)})$.
	\STATE New feature matrix with higher order terms:  $\tilde{\mathbf{X}} = [X_{\mathcal{S}},\text{Int}(X_{\mathcal{S}} ) )]$.
	\STATE Estimating performance: $\hat{\z}^{(k)} = \tilde{\mathbf{X}}^{(k)}\boldsymbol{\beta}^{(k)}, $ where $\boldsymbol{\beta}^{(k)} = \text{EN}(s(\z^{(k)}), s(\mathbf{\tilde{X}}^{(k)})$
\RETURN Slowdown: $\hat{\z}^{(k)}$.
\end{algorithmic}
\label{alg:SYSTEM}
\end{algorithm}

\PUNT{\TODO{Need a description of the algorithm here. We can't just put it
  in with no reference in this section.  Why is it here and what does
  it do?  What does complete design matrix mean as an input?  Why
  doesn't it return anything?  Basically, you need a paragraph that
  walks through the algorithm, what is its goal?  What are the inputs?
  What is the output?  What is it doing in English?}}

Algorithm \ref{alg:SYSTEM} summarizes \SYSTEMESP{}, which predicts
slowdown when $k$ applications are co-scheduled. The algorithm takes a
few samples of random combinations of applications running together,
denoted by $\left( s(\z^{(k)}), \; \x^{(k)} = s(\mathbf{X}^{(k)})
\right)$, a design matrix containing the low-level-features for all
${n \choose k}$ combinations (denoted by $\mathbf{X}^{(k)}$) and
outputs a slowdown vector for all combinations of applications. The
first step does a feature selection on the linear model using
Elastic-net as $\boldsymbol{\beta}^{(k)} = \text{EN}(y^{(k)},
\x^{(k)})$. The non-zero coefficients of $\boldsymbol{\beta}^{(k)}$
indicate the selected features, denoted by $\mathcal{S}$. Then, we
construct a higher-order feature matrix $\tilde{\mathbf{X}}$ for those
selected features only. We run elastic net again for the new feature
space to obtain the final model which makes the performance
prediction, denoted by $\hat{\z}^{(k)}$.



Algorithm \ref{alg:SYSTEM} is run entirely offline.  It produces the
slowdown estimates for all applications (even those not sampled) in up
to $k$ co-scheduling groups.  These slowdown estimates can then be
used to predict performance for new combinations of applications in
online schedulers.  We show examples in the next section.  We also
show how the estimates can be trivially updated as new measurements
become available online.
