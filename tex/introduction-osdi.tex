\section{Introduction}
% Running applications together
Applications co-scheduled on the same physical hardware
\emph{interfere} with one another by contending for shared resources
\cite{dwyer2012practical,kambadur2012measuring,Bubble-flux,merkel2010resource}.
We quantify interference as \emph{slowdown}, or the performance loss
one application experiences in the presence of co-scheduled
applications.  By accurately predicting this interference, a scheduler
can determine optimal assignments of applications to physical
machines, leading to higher throughput in batch systems and better
quality-of-service for latency-sensitive applications.

% Many, many instances of co-scheduled applications -- should use
% these to predict behavior, but it is hard
Data center and super computer operators often have a great deal of
accumulated data about past jobs and their interference, yet turning
this data into effective interference predictors is difficult
\cite{kambadur2012measuring}.  To apply machine learning to build an
accurate predictor from this data, two fundamental decisions must be
made: (1) what \emph{features} should be measured and (2) what
\emph{model} maps these features into an accurate prediction.  Smaller
feature spaces provide more computationally efficient models, but
may miss key data and reduce prediction accuracy.  The art to modeling
is managing the tradeoffs between feature set size and the model
accuracy.  One family of machine learning
techniques---\emph{regularization}--- addresses the particular problem
where the number of features is much larger than the number of
samples; \ie the problem is \emph{ill-posed} and unsolvable with
standard regression analysis.  Regularization methods solve such
ill-posed problems by simultaneously selecting both the features and
the model
\cite{hoerl1988ridge,tibshirani1996regression,zou2005regularization}.

This work explores such state-of-the-art regularization models for
predicting application interference.  We find that regularized linear
regression methods require a relatively small number of features, but
produce inaccurate models.  In contrast, non-linear models that
include \emph{interaction terms}---\ie permit features to be
multiplied together---are more accurate, but are extremely inefficient
and not practical for online scheduling.
%Practicality is
%a central concern as several statistical methods for predicting
%application interference have proven not to scale beyond two
%applications \cite{Kambadur2010}; \ie they are impractical to
%use when co-scheduling three or more applications.

% Our approach -- split feature selection and model building into two
We therefore combine linear and non-linear approaches to produce
accurate and practical predictions.  We call our approach ESP for
\textbf{E}stimating co-\textbf{S}cheduled \textbf{P}erformance.
\SYSTEMESP{}'s key insight is to split regularization modeling into two
parts: \emph{feature selection} and \emph{model building}.  \SYSTEMESP{}
uses linear techniques to perform feature selection, but uses
quadratic techniques for model building.  The result is a highly
accurate predictor that is still practical and can be integrated into
real application schedulers.

\SYSTEMESP{} assumes there is a known (possibly very large) set of
applications that may be run on the system and some offline
measurements have been taken of these individual applications.
Specifically, \SYSTEMESP{} measures low-level hardware features like
cache misses and instructions retired during a training phase.  At
runtime, applications from this set may be launched in any arbitrary
combination.  The goal is to efficiently predict the interference (\ie
slowdown) of co-scheduled applications.
% A secondary goal is to support scalability of co-scheduled
% applications so that more than two applications can share the same
% node.

% Use approach to build schedulers for single and multinode system
We evaluate \SYSTEMESP{} by integrating it into both single and
multi-node schedulers running on Linux/x86 servers.  In the
single-node case, we construct a batch scheduler that orders
application execution to minimize the total completion time.  In the
multi-node case, we build a first-come-first-serve scheduler that
assigns applications to nodes as they arrive to minimize application
slowdown.  We compare \SYSTEMESP{}-based schedulers to prior scheduling
techniques that use contention-aware heuristics to avoid interference
\cite{resense,merkel2010resource,Merlin}.  We also compare \SYSTEMESP{}'s
accuracy to predictors built with a number of cutting-edge
regularization methods. We find:
\begin{itemize}
\item The single-node \SYSTEMESP{} schedules are, on average, 27\% faster
  than techniques based on heuristics.  Even with its runtime
  overhead, the \SYSTEMESP{} results are only 5\% worse (on average) than
  an oracle that has perfect knowledge of interference and no
  overhead. See \secref{sched_single_proc}.
\item The multi-node \SYSTEMESP{} schedules are 60\% faster than activity
  vector based schedules and only 5\% to 13\% worse than an oracle.
  See \secref{sched-multi-node}.
\item Critically, \SYSTEMESP{} produces better results as more
  applications are scheduled.  \SYSTEMESP{} produces quantifiable
  performance predictions while heuristic techniques simply produce a
  binary decision: co-schedule or not.  \SYSTEMESP{}'s quantifiable
  predictions allow schedulers to make optimal decisions even when
  interference cannot be avoided. In contrast, heuristic techniques do
  not quantify interference and thus cannot rank decisions.  As the
  number of applications increases, the chance of heuristics making a
  very poor choice in the face of unavoidable contention also
  increases.  See \secref{exp:multi-node-job}.
\item \SYSTEMESP{} is more accurate than existing linear regression
  techniques in most cases.  When considering two applications,
  \SYSTEMESP{} is similar to existing regularized linear regression
  techniques.  Considering more than two applications, however,
  \SYSTEMESP{} is uniformly more accurate than standard linear regression
  techniques.  See \secref{st_model}.  For reasons explained in
  \secref{est-intro}, existing regularized regression methods with
  interaction terms cannot be evaluated for accuracy because their
  models are too complex to be implemented in practice.
\end{itemize}

% Contributions:
This work makes the following contributions:
%\begin{itemize}
\begin{inparaenum}[1)]
\item \SYSTEMESP{}, a regularization method for predicting application
  interference,
\item Demonstration of \SYSTEMESP{} in both single and multi-node
  schedulers,
\item Comparison to existing heuristic techniques on real systems,
\item Comparison of \SYSTEMESP{}'s predictive accuracy to other
  regularization methods,
\item Open-source code release\footnote{The code is available at
    https://github.com/ESPAlg/ESP}.
%\end{itemize}
\end{inparaenum}
\SYSTEMESP{} helps scheduling at multiple levels: it determines which
applications should run together on a single node, it determines which
node a new application should be scheduled on, and it avoids
disastrous decisions that heuristic schedulers cannot.  Support for
data analytics has become an important research topic in computing
systems, and this work explores how data analytics can be used to
improve computing systems.
