\section{Related Work}
\label{sec:related}
We discuss related work on energy and power optimization. Offline optimization techniques have been proposed (\eg \cite{Yi2003,LeeBrooks2006,CPR,ChenJohn2011,petabricksStatic}, but
they are limited by reliance on a robust training phase.  If behavior
occurs online that was not represented in the training data, then
these approaches may produce suboptimal results.

Several approaches augment offline model building with online
measurement.  For example, many systems employ control theoretic
designs which couple offline model building with online feedback
control
\cite{Wu2004,TCST,Chen2011,PTRADE,Heartbeats2,ControlWare,Agilos,Rajkumar,Sojka,Raghavendra2008}.
Over a narrow range of applications the combination of offline
learning and control works well, as the offline models capture the
general behavior of the entire class of application and require
negligible online overhead.  This focused approach is extremely
effective for multimedia applications
\cite{grace2,flinn99,flinn2004,xtune,TCST} and web-servers
\cite{Horvarth,LuEtAl-2006a,SunDaiPan-2008a}.  The goal of \SYSTEMLEO{},
however, is to build a more general framework applicable to a broad
range of applications.  \SYSTEMLEO{}'s approach is complementary to
control based approaches.  For example, incorporating \SYSTEMLEO{} into
control-based approaches might extend them to other domains even when
the application characteristics are not known ahead of time.

Some approaches have combined offline predictive models with online
adaptation
\cite{Zhang2012,packandcap,Winter2010,dubach2010,Koala,Cinder, wu2012inferred}.  For
example, Dubach et al.  propose such a combo for optimizing the
microarchitecture of a single core \cite{dubach2010}.  Such predictive
models have also been employed at the OS level to manage system energy
consumption \cite{Koala,Cinder}. \cite{wu2012inferred}.

Other approaches adopt an almost completely online model, optimizing
based only on dynamic runtime feedback
\cite{Li2006,Flicker,ParallelismDial,Ponamarev,petabricksDynamic,LeeBrooks}.
For example, Flicker is a configurable architecture and optimization
framework that uses only online models to maximize performance under a
power limitation \cite{Flicker}.  Another example, ParallelismDial,
uses online adaptation to tailor parallelism to application workload.


Perhaps the most similar approaches to \SYSTEMLEO{} are others that
combine offline modeling with online model updates
\cite{ICSE2014,Bitirgen2008,Ipek}.  For example, Bitirgen et al use an
artificial neural network to allocate resources to multiple
applications in a multicore \cite{Bitirgen2008}.  The neural network
is trained offline and then adapted online using measured feedback.
This approach optimizes performance but does not consider power or
energy minimization.

Like these approaches, \SYSTEMLEO{} combines offline model building and
with online model updates.  \emph{Unlike prior approaches, \SYSTEMLEO{}
  learns not a single best state, but rather all Pareto-optimal
  tradeoffs in the power/performance space (like those illustrated in
  \figref{fig:pareto})}.  These tradeoffs can be used to maximize
performance or to minimize energy across an application's entire range
of possible utilization.  There is a cost for this added benefit:
\SYSTEMLEO{}'s online phase is likely higher overhead than these prior
approaches that focus only on maximizing performance.  In that sense,
however, these approaches complement each other.  If fastest
performance is the goal, then prior approaches are likely the best
option.  If the goal is to minimize energy for a range of possible
performance, then \SYSTEMLEO{} produces near optimal energy.
